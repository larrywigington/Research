\documentclass[thesis,ieee,twoside]{npsreport2018}
%%%%%%%%%%%%%%%%%%%%%%%%%
% \documentclass Options
%%%%%%%%%%%%%%%%%%%%%%%%%
%
% Welcome to the NPS thesis template!  The "\documentclass" command above is
% where you tell LaTeX what type of document you would like to make.  We describe
% the options to you here.
%
% First note that the command has two parts, the options in square brackets [] and
% the name of the class, "npsreport2018", in the curly braces {}.
%
% thesis, dissertation, article, techreport:  The thesis, dissertation, and techreport
%     options enable multiple chapters and extend the LaTeX report class.  The article
%     option has no chapters (it is all one chapter) and extends the LaTeX article class.
% ieee or informs:  Select one of these options to set the citation style
% twoside:  Use this option for theses and disserations.  It forces all chapters to start
%     on odd-numbered pages (the right-hand side of a book) instead of the next page.
%     Omitting this option is recommended for other types.
% twoauthors, threeauthors, fourauthors:  The default is to have only one author so add
%     one of these options for specifying a number of authors greater than one
% acronym:  This option limits the acronyms printed in the acronym list (if you print one
%     at all) to those that you used in your document by activating the "printonlyused"
%     option of the acronym package.  Without this option, when you print the list of
%     acronyms, it will include all those defined, regardless of whether they are used
%     in the document.
%
% Advisors and Readers for thesis type:
%   twoadvisors:  two advisors but no second reader
%   twoadvisorsreader:  two advisors AND also a second reader
%   tworeaders:  two second readers
%
% Dissertation Committee Members for dissertation type:
%   fivemembers:  default committee size is six
%   advisoralone:  lists the top person alone
%
% Less Commonly Used Options:
%   classified:  use if you are in the STBL or SCIF writing classified work
%   singlespace:  as opposed to the standard spacing for a thesis
%   compacttitle
%   10pt, 11pt, 12pt:  sets font size, thesis requirement is 12pt
%   times, arial, courier:  sets the font face, thesis requirement is times
% 

%%%%%%%%%%%%%%%%%%%%%%%%%
% Packages
%%%%%%%%%%%%%%%%%%%%%%%%%
%
% LaTeX's extensibility comes through users' ability to use "packages."
% Here are some packages specifically here for use in this template.  Feel free to delete
% or comment out the first two of these:
\usepackage{hologo,lipsum} % provides \BibTex and \lipsum macros, for demos; can delete this package (just filler text)
\newcommand{\BibTeX}{\hologo{BibTeX}} %provides \BibTeX macro for demos; can delete this command
\ifnpstechreport
\else
\usepackage[fontsize=20pt,text=To~be~replaced~by~your~thesis~processor~using~the~data~from~your~Python~Thesis~Dashboard,color=red]{draftwatermark}
\fi

% For Example: you might find one of these useful:
\usepackage[defaultlines=2,all]{nowidow}    % automatic management of widows and orphans
\usepackage{listings}        % Use this for showing blocks of computer code
\usepackage{longtable}       % Use this for tables that are longer than one page
\usepackage{pdflscape}       % provides landscape environment
% \usepackage{stocksize}       % provides the ability to change the size of the paper  %%% Requires TeXLive 2025 or later!
%\usepackage{bibunits}        % Use this if you need to put references in your Executive Summary
%\usepackage{colortbl}        % provides commands to color table columns, rows, cells, etc.
%\usepackage{epstopdf}        % to use .eps files for figures
%\usepackage{bm}              % bold math if you need bold greek letters
%\usepackage{glossaries}      % comment out {doc,lipsum} package to use this package; see http://en.wikibooks.org/wiki/LaTeX/Glossary
%\usepackage{asymptote}       % for graphics
% The asymptote package allows for very nice graphics and figures
% Proper usage requires additional information located at:
% http://asymptote.sourceforge.net/
% See the gallery at this URL for examples

%\usepackage{placeins}        % float placement
% Provides \FloatBarrier which keeps figures/tables in the same section.
% LaTeX sometimes moves them to fill up pages.
% http://ftp.math.purdue.edu/mirrors/ctan.org/macros/latex/contrib/placeins/placeins-doc.pdf

%\usepackage[numbered]{mcode} % matlab code
% The mcode package must be separately downloaded.
% http://www.mathworks.com/matlabcentral/fileexchange/8015-m-code-latex-package

%\usepackage{flafter}         % float placement
% Ensures that figures/tables do not appear in the document before
% they are referenced in the text.

%%%%%%%%%%%%%%%%%%%%%%%%%
% Information about your work
%%%%%%%%%%%%%%%%%%%%%%%%%
%%%%%%%%%%%%%%%%%%%%%%%%%
% Information about your work
%%%%%%%%%%%%%%%%%%%%%%%%%
%
% LaTeX uses the following information about your work to create the "front matter" of
% your document.

\title{[Title]}

% Student info
\author{[Author Name]}
\rank{[Rank, Service]}    %\rank{Civilian} % if you don't have a rank
\degree{Master of Science in [Degree]}
\degreeabbreviation{MS}   % Should be MS, MBA or MA
\prevdegrees{[B.S., My Old School, Year]} % previous degree
% For students with multiple degrees, place \\ between each.

% Department info
\department{Department of [Department]}
\thesisadvisor{[Primary Advisor]}
\secondreader{[Second Reader]}
\departmentchair{[Department Chair]}

% The date you are graduating:
\degreedate{[Month Year]}

% Use same distribution statement selected on your thesis dashboard.
\distribution{Approved for public release. Distribution is unlimited.}

% Paste abstract from your thesis dashboard. New paragraphs start after an empty line.
\abstract{%
\lipsum[1] % example text, remove me
}

% If CUI (Controlled Unclassified Information), swap which line is commented-out (the % sign) for your drafts.
\securitybanner{}
%\securitybanner{CUI} % remove before requesting FINAL REVIEW with thesis processor, who will insert final banners.
%
% Do NOT upload CUI to public websites like Overleaf.com.
%
% If you are interested in getting a SECURE Overleaf site approved for NPS CUI work, visit https://nps.edu/web/thesisprocessing/latex-cui.
%
% Mandatory fields for the SF298.
%
\ifnpstechreport
    \ReportType{Technical Report}
\else
    \ReportType{Master's Thesis}
\fi
\ReportDate{[Month Year]}       % for a thesis, graduation date
\SponsoringAgency{N/A}          % really, for technical reports
\DatesCovered{MM-DD-YYYY to MM-DD-YYYY}
\ReportClassification{Unclassified}
\AbstractClassification{Unclassified}
\PageClassification{Unclassified}
%
% Optional fields for the SF298.
%
\RPTpreparedFor{}
\ContractNumber{}
\GrantNumber{}
\ProgramElementNumber{}
\TaskNumber{}
\WorkUnitNumber{}
\POReportNumber{}
\Acronyms{}
\SMReportNumber{}
\SubjectTerms{}
\ResponsiblePerson{}
\RPTelephone{}
\SignatureOne{}
\SignatureTwo{}
\SupplementaryNotes{The views expressed in this document are those of the author and do not reflect the official policy or position of the Department of Defense or the U.S. Government.
}

% Optional. Prevents footnotes from being reset at each chapter
% Comment this out to have them reset with each chapter.
\makeatletter
\@removefromreset{footnote}{chapter}
\makeatother

% Optional. Adds pdf metadata and links.
% This should be right before the \begin{document}, to be the
% last package / macros defined. (Hyper-ref is fragile,
% needs to be last, and has known conflicts with other packages.)
% Comment out if you have build problems building with hyperref
\NPShyperref


%%%%%%%%%%%%%%%%%%%%%%%%%
% Your thesis begins here.
%%%%%%%%%%%%%%%%%%%%%%%%%

\begin{document}
%% IEEE format only and Mechanical and Aerospace Engineering (MAE) Department only:
%% Uncomment the next line.  All others, leave commented out (with % sign)
%\bstctlcite{forced-et-al-off}
\NPScover                  % Cover page
\NPSsftne                  % SF298 form
%\NPSsignature             % Tech Report page (iii): signature page
\ifnpstechreport
    \else
    \NPSthesistitle            % Thesis page (iii): title page
    \fi
\NPSabstractpage           % Abstract Page
\NPSfrontmatter            % NPS front matter follows

\ifnpstechreport
\else
\DraftwatermarkOptions{stamp=false} % Starting at this point in the document, this commands the draftwatermark package to stop printing.
\fi

% This changes the chaptermark and includes the various tables
% It must be here.  Comment this line out for articles that do not contain chapters.
\renewcommand{\chaptermark}[1]{\markboth{\MakeUppercase{\chaptername}\ \thechapter.\ #1}{}}

%
% If you don't need one of these, comment it out.
%
%\setcounter{page}{7} % -- if you need to change the page number for the TOC do it here
\NPStableOfContents
\NPSlistOfFigures
\NPSlistOfTables
\NPSlistOfAcronymsFromFile{acronyms}

%
% Put (optional) Executive Summary here.
% New paragraphs start after an empty line. See exec_sum_with_refs.tex under the examples directory for a demo.
% If not using, comment-out or delete all of these lines.
%
\NPSexecsummary{%
This is different from your abstract. An executive summary is required or recommended for the departments listed \href{https://nps.edu/documents/105790666/106471207/Executive_Summary_Guidance.pdf}{\underline{here}}. Most executive summaries range from 2--5 pages.

An executive summary is a highly condensed version of your thesis that should be able to stand alone, independent of your thesis. An executive summary should summarize your purpose, methods, results, conclusions, and recommendations to allow someone who can read ONLY that document to walk away with a solid understanding of the research.

If you include figures or tables in your executive summary, do not use the caption commands for the titles. Instead, manually type in the titles. This will keep these titles out of your thesis's Lists of Figures and List of Tables, and allow the figure and table numbering to start at “1” in the thesis body, as required.

% How to have References in your Executive Summary:
If you include citations in the Executive Summary, it may make more sense to type your entries in manually, since most executive summaries contain only a handful of sources. To add a reference list programmatically, see exec\_sum\_with\_refs.tex in the ``additional\_resources'' node of this template for the code.
% Instead of having this sample Executive Summary content here, put your executive summary in the
% file additional_resources/exec_sum_with_refs.tex and use this next line to link it here:
%\begin{bibunit}[\NPSbibStyle]
%
% Put Executive summary here.
% New paragraphs start after an empty line.
% For the sake of this template, we use either the \citep{} or \citet{} commands but need to change all of them here to the \cite{} command for IEEE.  See the References section Chapter 1 for more information.
\ifinforms
\else
\let\citep\cite
\let\citet\cite
\fi
%

This is an example of how to create an executive summary with its own references 
section using the \texttt{bibunit} package.
The build process needs to change to accomodate this. 
The \texttt{bibunit} package builds separate unit files
\texttt{bu1.aux}, \texttt{bu2.aux}, etc. 
These needs to be run through \BibTeX{} separately.
In this example, the executive summary is the first (and only) bibunit, 
so we need to do the commands:
\begin{itemize}
	\item[] \texttt{pdflatex report}
	\item[] \texttt{bibtex report}
	\item[] \texttt{bibtex bu1}
	\item[] \texttt{pdflatex report}
	\item[] \texttt{pdflatex report}
\end{itemize}
The \texttt{Makefile} demonstrates how to script this.

\section*{Executive Summary Section} % Use the stared section/subsection/etc. commands here so they do not create TOC entries
The references~\citet{americans_1991,haynes_2009}
are in both the executive summary
and in the main thesis; they are numbered separately.
Some references~\citep{Unicorn:1995} appear only in the executive summary,
and are not part of the final List of References of the thesis.

Note, you cannot use numbered sections in the executive summary,
since the summary has no number itself.

This sentence demonstrates that acronyms, like \ac{US} and \ac{TCP}, work in the
executive summary; the \ac{US} is the short version. The counter will be re-set
in the main body, where its first use will be long again.

\subsection*{An Exec Summary Subsection}
\lipsum[2-3] % example text; remove me

%
% this makes references appear here at the subsubsection-level instead of chapter-level
%
\begingroup
 \let\stdthebibliography\thebibliography
 \renewcommand{\thebibliography}{%
 \let\chapter\subsubsection
 \titlespacing*{\subsubsection}{0pt}{5ex plus 2ex}{-1ex plus .2ex}
 \stdthebibliography}
 \raggedright     % don't automatically full justify bibliographic references
 \singlespacing   % reduce extra line breaks between entries
 \normalsize
 \putbib[references]  % use references.bib and place the bibliography here
\endgroup
%
% This is the end of special macros that tweak the appearance of the references
%
\end{bibunit}

}

%
% Put (optional) acknowledgements here.  
% New paragraphs start after an empty line.
% If not using, comment-out or delete all of these lines.
%
\NPSacknowledgements{%
Optional for all departments. Acknowledgments may be more informal in tone than the main thesis text but should still follow the correct use of sentence structure, grammar, and punctuation. 

Write your own acknowledgments; do not borrow the text from another author. While acknowledgments tend to be similar across document types, they should reflect your own language and sentence structure.
}

% Start layout for the NPS body
\NPSbody


%%%%%%%%%%%%%%%%%%%%%%%%%
% Here's where you put the actual content for the body of your thesis.
%%%%%%%%%%%%%%%%%%%%%%%%%

% CHAPTERS
% You have two options on how to structure your thesis:
% a) A single file. All chapters, sections, etc. go in this file.
%    This can make navigating your thesis a little more difficult.
% b) Use multiple files.  One chapter per file is recommended.
%    This breaks your thesis up into logical units to edit.
%
\chapter{Introduction}\label{ch:common}

This is the beginning of Chapter~\ref{ch:common}. 
Always have text between every head and subhead. You have a lot of control over the placement of your figures and tables. Examine the source files for the lines of code used to create the figures and tables exampled in this template. 

\section{A Bit of History}

This NPS \LaTeX{} thesis template was created in 2011 by faculty members in the Computer Science Department. NPS technical report NPS-CS-11-011 by Garfinkel and Axtell (find it in the NPS library!) provides an introduction to \LaTeX{} and the basic structure of the thesis template. However, this template has evolved since 2011 in a somewhat {\it ad hoc} manner, and some of the information in Garfinkel and Axtell's technical report is outdated.

\section{References}
The reference style is the most noticeable difference between the IEEE and INFORMS styles.
The NPS thesis template has \LaTeX{} use the \BibTeX{} engine to build the references list.
All references go into a .bib file.  This template uses references.bib.  Only the references that you cite in your document will appear in the references list.  This enables you to use one .bib file for multiple projects.

\subsection{Citing References}
To cite your references, you create a reference key for each source entry in your .bib file.  For example, if your \BibTeX{} entry is of the form {\tt @article\{Jones:1989,\dots}, then you can cite this reference with the key {\tt Jones:1989}.
\begin{itemize}
    \item \textbf{IEEE} users need to use the \verb|\cite{[key]}| command.
    \item \textbf{INFORMS} users need to use either the \verb|\citep{[key]}| command for a parenthetical citation or \verb|\citet{[key]}| for an author-in-text citation.
\end{itemize}

% For the sake of this template, we use either the \citep{} or \citet{} commands but need to change all of them here to the \cite{} command for IEEE:
\ifinforms
\else
\let\citep\cite
\let\citet\cite
\fi

\subsubsection{Start with [1] for IEEE!}\label{sec:firstone}
The references should begin with citation \citep{pollan_2006} in your main thesis body. If they start with another number, it is most likely caused by citation numbers in your figure captions or table titles, which appear ``first'' in the List of Figures and List of Tables, since these appear before your first chapter. To fix this, use a \underline{two-part caption}; see the caption for Figure \ref{fig:dragon}. If references in the executive summary are causing it, see how to use the ``bibunit'' environment in the ``exec\_sum\_with\_refs.tex'' file in the ``additional\_resources'' folder of this template.

For a couple tips on citing in the IEEE style with \LaTeX{}:
\begin{enumerate}
    \item When citing multiple references as a group, \LaTeX{} delineates them with commas, even if they appear consecutively in the references list~\cite{pollan_2006,Crabtree:Chaplin:2013,DOD.8570.01-M}.
    \item Sometimes you will notice a \verb|~| preceding a \verb|\cite| command in the source of this template.  The \verb|~| acts as a non-breaking space.  It is optional, but can be helpful to prevent a citation's number from word wrapping to its own line.
\end{enumerate}

\subsubsection{How to Cite in Text Using INFORMS Style}
See the SETUP file for instructions on switching this template to INFORMS style. There are several ways of citing references. \textbf{If you still have this template configured for IEEE, some of these examples will look wrong.}
\begin{enumerate}
    \item {\it With the author(s) as a noun (e.g., the subject) in a sentence:}
    In \citet{pollan_2006}, renewed interest was expressed in eating anything and everything, even if in an {\it ad hoc} manner; see \citet{Crabtree:Chaplin:2013} for a general treatment.  
    \item {\it As a parenthetic reference supporting a statement:} 
    An important contribution in the development of DOD information assurance policy is the connection to game theory \citep{DOD.8570.01-M}. 
    \item {\it A reference that is part of a longer parenthetic statement:} 
    Understanding the way in which pizza becomes a ``locally optimal'' strategy when weighed against other foods is a topic for future consideration \ifinforms\citep[see][for a discussion]{Yoshi:1988}\else\verb|\citep[see][for a discussion]{Yoshi:1988}|\fi.
    \item {\it Multiple citations:} 
    Tutorial material is available from several sources \citep{Monster:1985, Nekeip:2008, pollan_2006}
\end{enumerate}

\subsection{Example Library References}
The citation format approved by the Thesis Processing Office is shown online at \url{https://libguides.nps.edu/citation/bibtex}. This template samples each of its entries. Match the source type below to the entry in the List of References at the end of this template.

\begin{itemize}
    \item Blog:  \citep{locke_2020}
    \item Book Chapter (in edited book), one author, one editor:  \citep{haynes_2009}
    \item Electronic book:  \citep{bonds_2014,krishnan_2008,Crabtree:Chaplin:2013}
    \item Book (print), one author:  \citep{pollan_2006}
    \item Book (print), two authors:  \citep{strindberg_warn_2011}
    \item Book (print), three authors:  \citep{Cordesman:2009}
    \item Book (in series):  \citep{abramowitz_64}
    \item Book (portion):  \citep{orend_2013}
    \item Book (volume):  \citep{myer_77}
    \item Book Chapter (in edited book), three authors, two editors:  \citep{Cordesman:2009}
    \item Class Notes / Lecture, Published:  \citep{Blanche:2017}
    \item Class Notes, Unpublished:  \citep{Houston:2016}
    \item Lecture, Unpublished:  \citep{Norton:2014}
    \item Presentation or Workshop:  \citep{Horse:2017}
    \item Computer Program / Software, online:  \citep{comprehensive_2005}
    \item Conference Proceedings (online):  \citep{morentz_2009}
    \item Conference Proceedings (print):  \citep{katz_2007}
    \item Paper Presented at Conference, Unpublished:  \citep{Teplin:EtAl:2005}
    \item Data Set / Database, Published:  \citep{Suro:2004,nsa_ipac_2012}
    \item Data Set, Unpublished:  \citep{Blanche:2006}
    \item Dictionary / Encyclopedia:  \citep{merriam_2017}
    \item Fact Sheet:  \citep{FLSA:2008}
    \item Directive / Instruction:  \citep{DOD.8570.01-M}
    \item Memorandum \citep{takai_2013}
    \item Joint Doctrine:  \citep{JP-3-01}
    \item Field Manual:  \citep{sniper_2011}
    \item CRS or GAO Report:  \citep{erwin_2011}
    \item Handbook (online):  \citep{TSP-168:1972}
    \item Handbook (print):  \citep{transmission_comm_85}
    \item Journal Article (online):  \citep{sanico_2018}
    \item Journal Article (print):  \citep{Griffin:2009}
    \item (IEEE only, citation not required for INFORMS) Public Law:  \citep{americans_1991}
    \item Map:  \citep{Google_2017}
    \item Multimedia/Video:  \citep{youtube_2014}
    \item Magazine / Newspaper Article:  \citep{linguine_2016}
    \item Magazine/ Newspaper Article, author known:  \citep{Beforebad:2014}
    \item Patent:  \citep{bell_1876}
    \item Personal Communication / Email:  \citep{Wunkerbunk:2002}
    \item Interview:  \citep{Monster:1985}
    \item Report / Technical Report / Working Paper / think tank:  \citep{dixon_2017,Wonka:1972}
    \item Secondary Source:  \citep{Nicholson:2003}
    \item Standard:  \citep{standard_1968}
    \item Dissertation from school's archive such as Calhoun:  \citep{Yoshi:1988}
    \item Thesis from commercial database:  \citep{Nekeip:2008}
    \item Unpublished Work Accepted for Publication:  \citep{Horse:1996} 
    \item Unpublished Work (print):  \citep{Horse:1995} 
    \item Unpublished Work (arXiv):  \citep{simonyan-vgg16-2015}
    \item Webpage, author and publication date given:  \citep{Sushi:1995}
    \item Webpage, no author given, organization as author:  \citep{FBI_2017}
    \item Webpage, no date given:  \citep{Python:2017}
    \item Webpage, Janes example:  \citep{Janes_2017}
    \item Wikipedia: \citep{wiki_2016}
    \item Working Paper:  \citep{sushi_2021}    
\end{itemize}

\subsection{Best Practices in the {\tt bib} File}

Refer to the {\tt references.bib} file for the coding used to produce entries exactly like those pictured on NPS's reference guide at \url{https://libguides.nps.edu/citation/bibtex}. Some of the tricks used are as follows:
\begin{itemize}
    \item \textbf {Underscores or \% in web addresses.} Underscores or \% in web addresses will corrupt the format of the entry, so always insert a backslash before each, like this: {\tt web\textbackslash\_address}, {\tt web\textbackslash\%address}
    \item \textbf {Capital words.} Use braces to retain word capitalization when needed, especially around acronyms and around proper nouns in titles, as in: {\tt \{The summer when \{Bob\} grew up\}}
    \item \textbf {Organization as author.} Use double braces around any organization names in the {\tt Author} field. This stops the {\tt bst} file from inverting it and reducing words to an initial, like it does to a person's name.
    \item \textbf {``U.S.'' error.} If ``United States'' is part of the {\tt author} name, use double braces: {\tt \{\{United States Navy\}\}}. \textbf{Do not use ``U.S.'' in the {\tt Author} field}. Instead, SPELL OUT as ``United States.'' The bst file is hardwired to remove periods, to invert, and to use the first initial, so ``U.S. Navy'' or ``US Navy'' will output as ``Navy U'' (not good).
    \item \textbf {Colons in titles.} If there is a colon in a title, note that \LaTeX{} will capitalize the first word after the colon.
    \item \textbf {Don't forget those commas!} There must be a comma after every field (except for the last field of each entry). Without the comma, the fields that follow will not output.
    \item \textbf {Recompile often}. Catch errors as you input data, rather than trying to catch errors by proofreading the entire list all at once (even the best eyes miss things). 
\end{itemize}

When creating a new \BibTeX{} entry, you do have several degrees of freedom in making things work:
\begin{itemize}
    \item You can choose different entry types (e.g., {\tt @manual} vs.~{\tt @misc}).
    \item You have flexibility in the choice and format of fields within each entry (e.g., the {\tt note} field is often a ``catch all'' for information that doesn't naturally fit in other fields). 
\end{itemize}

There is no absolute right or wrong in the creation of these entries. All that matters is the final product: how the entry looks in the final List of References. The guesswork for many entry types, however, has been done for you; refer to the {\tt references.bib} file for the techniques that result in entries matching the format approved by the Thesis Processing Office. 

\section{Sections}
\LaTeX{} offers multiple outline levels.
They are chapter, section, subsection, subsubsection, paragraph, subparagraph, and subsubparagraph.

Sometimes you need to have a shortened version a section's title appear in the table of contents.
To do this, use \verb|\chapter[]{}|, just like we use \verb|\caption[]{}| for the figure examples below.
The square brackets contain the TOC entry and the curly braces contain the actual chapter/section title.
You can also use this technique for keeping other macros, \textbf{like acronyms and footnotes}, out of the table of contents.


\section{Acronym Management}
This is a good way to manage acronyms throughout your thesis.
This shows the acronym macro being used for \ac{TCP}, which is produced in its short form \ac{TCP} on all subsequent uses of the macro.
Acronyms are reset after the abstract and executive summary, and will be shown in their long form in their first use by default.
The acronym package has a lot of capability to use acronyms, including a short form \acs{TCP} and plural form \acp{SRWBR}; see the acronyms file and acronym package documentation for additional details.
Here is another example using \ac{DOD} then \ac{DOD} or \ac{NPS} then \ac{NPS}.
Finally, \ac{MASINT} as later appears as \ac{MASINT} and \ac{USN} likewise appears subsequently as \ac{USN}.
You can read the \texttt{acronym} package documentation to learn more about other features, including how to have a different indefinite article preceding your acronym, appropriate for whether \LaTeX{} renders the long or short form.

\section{Figure Formatting}\label{sec:figures}

In the source code file, take a look at the code to produce Figure \ref{fig:dragon} and its placement.
\begin{itemize}
    \item Use an [H] float to place the figure exactly at that spot in your document, and to prevent figures from landing in the middle of paragraphs. 
    \item Use a \verb+\vspace+ to add space between each figure and the text above it. 
    \item You may also want or need to change the image size. Image scaling is also shown in Figure \ref{fig:dragon}.  
    \item Refer to each figure by its number in the body text. Examine the source file in this section for how to label and then cross reference your figures.
\end{itemize}

Figures should be readable if the words in them are meant to be read. You may need to re-create images when the source text is too fuzzy to read. Each figure must be referred to by its number in the body text.
See Section~\ref{sec:uncommon-figures} for examples of less common figures.

\begin{figure}[H] % use an H float to place the figure at that specific spot, and to prevent a figure from landing in the middle of a paragraph.
    \centering
    \includegraphics[scale=.7]{figs/dragon.jpg}
    \caption[Short caption for List of Figures only; full caption in main text.]{Compare this caption to its match in the List of Figures. The caption displays in full here, but only the text enclosed in the square brackets appears in the List of Figures. This is how to have longer captions, but a succinct list of figures. This method also prevents citations from appearing in the LoF and, for IEEE, ensuring the first citation will start at ``1'' in the main text.  Source:~\citet{pollan_2006}.}
    \label{fig:dragon} % for cross referencing
\end{figure}

You can also use figures to display the source code for computer programs or scripts that you created to conduct your research.
The \LaTeX{} \verb|listings| package (\url{https://www.ctan.org/pkg/listings}) will format (and, with some additional configuration, color) your code according to the language.
See Figure~\ref{fig:python-short} for code that is shorter than one page and Section~\ref{sec:uncommon-figures} for code that is longer than one page.

\begin{figure}[H] % use an H float to place the figure at that specific spot, and to prevent a figure from landing in the middle of a paragraph.
    \centering
    \lstinputlisting[language=Python,numbers=left,breaklines,frame=single]{figs/Python-short.py}
    \caption{A Python Code Sample}
    \label{fig:python-short} % for cross referencing
\end{figure}

\section{Table Formatting}\label{section:tableformat}
See lines of code used to create Table \ref{table:sampletab} in the source document for format considerations regarding tables.      
     
\begin{table}[H] % H prevents table from landing in the middle of a paragraph
    \centering
    \caption[Table short title]{Table short title. Adapted from~\citet[table 5]{congress_1991}.} % note brackets and curly brackets to display citation here but not in the List of Tables; note also that this table is not entirely original and therefore requires an ``Adapted from'' in the table title.
    \label{table:sampletab}
    \begin{tabular}{ c c c }
    \hline
      1 & 2 & 3 \\ \hline
      4 & 5 & 6 \\
      7 & 8 & 9 \\
      10 & 11 & 12 \\
      13 & 14 & 15 \\
    \hline
    \end{tabular}
\end{table}

If you have more data to show in a table than will fit vertically on one page, consider using a long table, as demonstrated in Section~\ref{sec:uncommon-figures}.
Other options for more advanced tables include the \texttt{tabularray} package.

\section{Tips}
Following the guidance here as you draft your thesis will save you time later in going back to implement it.

\subsection{Examples}
Look at the source \LaTeX{} code to see how to render these parts.

\subsubsection{Footnotes}
Here we demonstrate footnotes. Be sure to end all footnotes with a period.\footnote{This is a sample footnote.} See the bottom of the page for the footnote.
Sometimes you need to use a footnote in a chapter or section title.
To do this, use \verb|\chapter[]{}|, just like we use \verb|\caption[]{}| for Figure~\ref{fig:dragon}.

Footnotes are also possible in tables, but you must consider whether you want a document-level footnote or a footnote that pertains just to the table, that is, a table note.  In most cases, table notes are most appropriate.

\begin{table}[H] % H prevents table from landing in the middle of a paragraph
    {\centering % For a table note, insert an opening brace { before the \centering command.
    \caption{A Table with a Footnote and Table Note}
    \label{table:sample}
    \begin{tabular}{ c c c }
    \hline
      7 & 8 & 9 \\
      4 & 5 & 6 \\
      1 & 2 & 3 \\
      0\footnotemark &  & . \\
    \hline
    \end{tabular} \par} % Insert "\par }" to tell LaTeX you want another paragraph after this (but still within the table float environment) and then provided the closing brace to end the portion that you want centered.
    {\footnotesize\textit{Note:} Here is my note about something in the table.  You will need all the code in this line.  This table looks like a keyboard's number pad.\singlespacing} % If your note ends up being only one line long, then you may get an error when trying to use \singlespacing, in which case you will need to omit the \singlespacing command.
\end{table}
\footnotetext{This is for a footnote with broader scope than is appropriate for a table note.  Zero is an important number.}

\subsubsection{Equations}
A simple equation to check numbering.
\begin{equation} \label{eq:pi-theory}
a^2 + b^2 = c^2
\end{equation}

Another equation that follows Equation \ref{eq:pi-theory}:
\begin{equation}
    L' = {L}{\sqrt{1-\frac{v^2}{c^2}}}
\end{equation}

When an equation occurs in the middle of a sentence, such as this one involving $e \in \mathbb{R}$,
\begin{eqnarray}
 e^x &\approx& 1+x+x^2/2! \nonumber \\
   && \hphantom{1} + x^3/3! + x^4/4! \nonumber \\
   && \hphantom{1} + x^5/5!,
\end{eqnarray}
then we need proper punctuation (such as the comma above) and the sentence ends here, on the next line.

\subsubsection{Quotes}
Here is an example of a block quote:
\begin{quote}
    \lipsum[2] \citep{katz_2007}  
\end{quote}

To output curly (``smart'') quotes, versus undesirable straight quotes ("like this"), use two tildes for the opening quote mark, and two apostrophes for the closing quote mark. There are many examples in this template, including in this paragraph.
Sometimes, a quote can exist within another quote.  To distinguish between the layers of quoting, you can use \verb|\,| to create a narrow space.  For example, ``He said, `Hello.'\,''

\subsubsection{Widows and Orphans}
Sometimes you will see just one line of a larger paragraph at the top or bottom of a page.  At the bottom of a page, it is called a widow, and at the top of a page, it is called an orphan; both are considered ``ugly.''
To ensure that your thesis has no widows or orphans, we have \LaTeX{} do the work for you with using this line in your preamble:
\begin{quote}\verb|\usepackage[defaultlines=2,all]{nowidow}|\end{quote}
\ldots or you can control them manually with the \verb+\pagebreak+ command. For more ideas, \href{https://wiki.nps.edu/pages/viewpage.action?pageId=720175227}{\underline{click here}}.


\subsection{Style and Formatting}
Please consider the following guidelines to ensure a smoother process with the Thesis Processing Office:
\begin{enumerate}
    \item Punctuation (periods and commas) go inside quotation marks. 
    \item The macros \verb+\etal+, \verb+\ie+, \verb+\eg+ and \verb+\etc+ force proper
      American English convention for these (\ie a comma follows).
      It is redundant and incorrect to use \etc at the end of a list of
      examples (\eg apples, pears, \etc).
    \item Chicago style recommends subordinate clauses using \verb+\ie+ and \verb+\eg+ be
    separated from the main clause using parentheses (\eg, as shown above).
    \item Use the \LaTeX{} \verb+\begin{figure}+ and \verb+\begin{table}+ environment to
      create floating figures and tables. Use the \verb+\caption+ command
      to create your captions. Label your captions with the
      \verb+\label{foo}+ command inside the caption itself. Reference
      these figures and tables with the \verb+\ref{foo}+ reference command.
    \item Do not split text around a figure or table. 
    \item Thesis Processing prefers periods in ``U.S.,'' for example, U.S. Navy.
    Common acronyms do not appear in the list of acronyms (\eg, \ac{US}, \ac{FBI}, \ac{CIA}).
    \item Master's degree has an apostrophe and Postgraduate is one word. 
    \item Most acronyms do not need periods, like \ac{NPS}. Common acronyms need not appear in the list of acronyms (\eg, \ac{US}, \ac{FBI}, \ac{CIA}).
    \item When typing a date, do not use ``st'' or ``th.'' Instead, just
      note the date: July 4, 1776, is Independence Day. Commas go 
    after month/date, year: both Jefferson and Adams died on July 4, 1826.
    No comma between month/yr: \textit{Alice's Adventures in Wonderland} was published in July 1865.
    \item In general, spell out numbers one through nine, and use numerals for 10 and greater.
    \item Capitalize C in Chapter, F in Figure, T in Table and E in Equation when referring
    to your own chapters, figures, tables and equations.
    \item When there is more than one reference, put them both into the \verb+\cite+ command: \verb+\cite{john1,john2}+. It will render like this \citep{sanico_2018,Griffin:2009,linguine_2016}.
    \item When typing equations in text, use ``where'' or ``if.'' Use Math Mode. 
    \item When inserting symbols, use Math Mode.
\end{enumerate}



% APPENDICES
% Appendices are not mandatory thesis components. You should use them if you need them.
% You have three options:
% 1) No appendix. Comment-out the lines in this section.
%
% 2) A single appendix (with a single TOC entry). Use these lines, adapted to your needs:
\NPSappendixTOC{[My Appendix Title]}
Sometimes it can be helpful to put extra information into an appendix.

\section{Less Common Figure Examples} \label{sec:uncommon-figures}
Here are a few example figure types that are less common.

\begin{figure}[H] % floats ask LaTeX to attempt to place figure here; used because fig originally landed on the next page, leaving big white space at the bottom of this page.
    \framebox[\textwidth]{\parbox{\textwidth}{\lipsum[65]}}
    \caption*{\small This is the long subcaption that explains the figure in detail and expounds on its relevance to the text. The other formatting option for long captions is shown in Figure \ref{fig:dragon}; choose one method for your thesis. All figures need to be referenced in the text before the image. Full source citation, as applicable, is required. Source:~\citet{Wunkerbunk:2002}.} \vspace*{1.5em}
    \caption{Figure caption, with \emph{emph} and \textit{italics} in a caption.}
\end{figure}

\begin{figure}[H] % H prevents figures from landing in the middle of paragraphs
    \framebox[\textwidth]{\parbox{\textwidth}{\lipsum[65]}}
    \caption{Some styled math in a caption, $\mathsf{Func}(x, \sigma) = x^2 + \overline{\sigma} \times \pi$.}
\end{figure}

Do you need to incorporate multiple subfigures within one figure?

\begin{figure}[H] % H prevents figures from landing in the middle of paragraphs
    \centering
    \subfloat[First sub-figure]{
       \framebox[0.47\textwidth]{\parbox{0.45\textwidth}{\lipsum[65]}}
    }
    \hfill
    \subfloat[Second sub-figure]{
       \framebox[0.47\textwidth]{\parbox{0.45\textwidth}{\lipsum[65]}}
    }
    
    \caption{Caption using subfigure package.}
\end{figure}

\subsection{A Long Code Listing}
As introduced in Section~\ref{sec:figures}, you can also use figures to display source code.
When your code listing is longer than one page, like the following example, it should have its own appendix or section in an appendix rather than being a figure.

\lstinputlisting[language=bash,numbers=left,breaklines,frame=single,aboveskip=0.6in]{figs/bash-long.sh}
% Use the `aboveskip=0.6in` parameter if you have text between the appendix title and the code listing.

\subsection{A Long Table}
As introduced in Section~\ref{section:tableformat}, you may need to show data in a table that is longer than one page, like Table~\ref{tab:long}.

\begin{center} % NB:  Unlike putting \centering within a table environment, the longtable environment goes inside the center environment
\begin{longtable}{lr}
    \caption{A sample long table showing decimal ASCII 55--90} \label{tab:long} \\ \hline
    Character & Number \\ \hline \endfirsthead % This caption and row appear at the top of the table
    \caption[]{A sample long table showing decimal ASCII 55--90} \\ \hline
    Character & Number \\ \hline \endhead % This caption and row appears at the top of each subsequent page
    % Very important:  The caption for each subsequent page must have [] to prevent the table from being listed multiple times (once for each page!) in the List of Tables.  Also, the label should appear only on the first instance of the caption.
    \hline \multicolumn{2}{r}{Continued on next page\ldots} \endfoot % This row appears at the bottom of all but the last page of the table
    \hline Total & 36 \\ \hline \endlastfoot % This row appears at the end of the table and is not always necessary.
    % You need to specify the above before providing your table data below.
    7 & 55 \\
    8 & 56 \\
    9 & 57 \\
    : & 58 \\
    ; & 59 \\
    < & 60 \\
    = & 61 \\
    > & 62 \\
    ? & 63 \\
    @ & 64 \\
    A & 65 \\
    B & 66 \\
    C & 67 \\
    D & 68 \\
    E & 69 \\
    F & 70 \\
    G & 71 \\
    H & 72 \\
    I & 73 \\
    J & 74 \\
    K & 75 \\
    L & 76 \\
    M & 77 \\
    N & 78 \\
    O & 79 \\
    P & 80 \\
    Q & 81 \\
    R & 82 \\
    S & 83 \\
    T & 84 \\
    U & 85 \\
    V & 86 \\
    W & 87 \\
    X & 88 \\
    Y & 89 \\
    Z & 90 \\
\end{longtable}
\end{center}

\subsection{A Wide Figure or Table}
If you need to present a figure or table that is too wide to fit between the margins, you can put in on a landscape page, as in Figure~\ref{fig:wide}.
Increasing the page size (in portrait or landscape layout) up to $11''\times17''$ is also permissible, but only in an appendix or supplemental.
For keeping this template simpler, we leave this larger page commented-out.
If you need this a larger page, feel free to un-comment these lines and use the example code as your template.

\begin{landscape}
\vspace*{\fill} % Use this vertical fill macro both above and below the figure/table to center it vertically on the page
\begin{figure}[H]
\centering
    \fbox{
        \includegraphics[width=0.45\linewidth]{figs/dragon.jpg}
        \includegraphics[width=0.45\linewidth]{figs/dragon.jpg}
    }
\caption{A Wide Figure}
\label{fig:wide}
\end{figure}
\vspace*{\fill}
\end{landscape}

% \newstocksize{layoutsize={11in,17in},keepmargins} % Using this requires the stocksize package, which requires TeXLive 2025 or later!
% \begin{landscape} % If you need a landscape page, keep the dimensions in the previous line portrait and use the landscape environment
% \vspace*{\fill} % Use this vertical fill macro both above and below the figure/table to center it vertically on the page
% \begin{figure}[H]
% \centering
%     \fbox{
%         \includegraphics[width=0.45\linewidth]{figs/dragon.jpg}
%         \includegraphics[width=0.45\linewidth]{figs/dragon.jpg}
%     }
% \caption{A Very Big Figure}
% \label{fig:really-big}
% \end{figure}
% \vspace*{\fill}
% \end{landscape}
% \restorestocksize

\section{Creating and Using Macros}
\LaTeX{} macros are commands that you can use as shortcuts to reproducing the same code.
For example, if you want to leave notes to yourself that contrast well with your thesis text, consider putting this macro definition in the main.tex file with your \verb|\usepackage{}| commands:

\verb|\newcommand{\jl}[1]{\textcolor{red}{JL: #1}}|

In this case, every time you put \verb|\jl{fix this!}| in your \LaTeX{} source, you will see \textcolor{red}{JL: fix this!} in your text.
The first parameter of \verb|\newcommand| is the name of the macro that you are defining.  In this case, I pretended my initials are JL.  You can use any name for a macro that is not already used.
The second parameter tells \LaTeX{} to expect one (1) parameter to this new macro.
The third parameter tells \LaTeX{} what to do when seeing this new macro.  In this case, it prints in red text: my initials, a colon, a space, and the text that I provide to the command.

If you need to have all of your comments disappear, for example, to submit your thesis draft to the TPO for its initial review, you can renew the command to hide all of your comments.  Just insert this line right after the above \verb|\newcommand| line:

\verb|\renewcommand{\jl}[1]{}|

As with nearly everything in \LaTeX{}, there is always a fancier way to do something.
In the case of making notes to remind yourself of something in a part of the text, the \texttt{todonotes} package offers a lot of features that you might find helpful.

%
% 3) Multiple appendices. Use this line:
%\NPSappendices
%    ...followed by this set of two lines for each appendix:
% \chapter{[First appendix title]}
% Sometimes it can be helpful to put extra information into an appendix.

\section{Less Common Figure Examples} \label{sec:uncommon-figures}
Here are a few example figure types that are less common.

\begin{figure}[H] % floats ask LaTeX to attempt to place figure here; used because fig originally landed on the next page, leaving big white space at the bottom of this page.
    \framebox[\textwidth]{\parbox{\textwidth}{\lipsum[65]}}
    \caption*{\small This is the long subcaption that explains the figure in detail and expounds on its relevance to the text. The other formatting option for long captions is shown in Figure \ref{fig:dragon}; choose one method for your thesis. All figures need to be referenced in the text before the image. Full source citation, as applicable, is required. Source:~\citet{Wunkerbunk:2002}.} \vspace*{1.5em}
    \caption{Figure caption, with \emph{emph} and \textit{italics} in a caption.}
\end{figure}

\begin{figure}[H] % H prevents figures from landing in the middle of paragraphs
    \framebox[\textwidth]{\parbox{\textwidth}{\lipsum[65]}}
    \caption{Some styled math in a caption, $\mathsf{Func}(x, \sigma) = x^2 + \overline{\sigma} \times \pi$.}
\end{figure}

Do you need to incorporate multiple subfigures within one figure?

\begin{figure}[H] % H prevents figures from landing in the middle of paragraphs
    \centering
    \subfloat[First sub-figure]{
       \framebox[0.47\textwidth]{\parbox{0.45\textwidth}{\lipsum[65]}}
    }
    \hfill
    \subfloat[Second sub-figure]{
       \framebox[0.47\textwidth]{\parbox{0.45\textwidth}{\lipsum[65]}}
    }
    
    \caption{Caption using subfigure package.}
\end{figure}

\subsection{A Long Code Listing}
As introduced in Section~\ref{sec:figures}, you can also use figures to display source code.
When your code listing is longer than one page, like the following example, it should have its own appendix or section in an appendix rather than being a figure.

\lstinputlisting[language=bash,numbers=left,breaklines,frame=single,aboveskip=0.6in]{figs/bash-long.sh}
% Use the `aboveskip=0.6in` parameter if you have text between the appendix title and the code listing.

\subsection{A Long Table}
As introduced in Section~\ref{section:tableformat}, you may need to show data in a table that is longer than one page, like Table~\ref{tab:long}.

\begin{center} % NB:  Unlike putting \centering within a table environment, the longtable environment goes inside the center environment
\begin{longtable}{lr}
    \caption{A sample long table showing decimal ASCII 55--90} \label{tab:long} \\ \hline
    Character & Number \\ \hline \endfirsthead % This caption and row appear at the top of the table
    \caption[]{A sample long table showing decimal ASCII 55--90} \\ \hline
    Character & Number \\ \hline \endhead % This caption and row appears at the top of each subsequent page
    % Very important:  The caption for each subsequent page must have [] to prevent the table from being listed multiple times (once for each page!) in the List of Tables.  Also, the label should appear only on the first instance of the caption.
    \hline \multicolumn{2}{r}{Continued on next page\ldots} \endfoot % This row appears at the bottom of all but the last page of the table
    \hline Total & 36 \\ \hline \endlastfoot % This row appears at the end of the table and is not always necessary.
    % You need to specify the above before providing your table data below.
    7 & 55 \\
    8 & 56 \\
    9 & 57 \\
    : & 58 \\
    ; & 59 \\
    < & 60 \\
    = & 61 \\
    > & 62 \\
    ? & 63 \\
    @ & 64 \\
    A & 65 \\
    B & 66 \\
    C & 67 \\
    D & 68 \\
    E & 69 \\
    F & 70 \\
    G & 71 \\
    H & 72 \\
    I & 73 \\
    J & 74 \\
    K & 75 \\
    L & 76 \\
    M & 77 \\
    N & 78 \\
    O & 79 \\
    P & 80 \\
    Q & 81 \\
    R & 82 \\
    S & 83 \\
    T & 84 \\
    U & 85 \\
    V & 86 \\
    W & 87 \\
    X & 88 \\
    Y & 89 \\
    Z & 90 \\
\end{longtable}
\end{center}

\subsection{A Wide Figure or Table}
If you need to present a figure or table that is too wide to fit between the margins, you can put in on a landscape page, as in Figure~\ref{fig:wide}.
Increasing the page size (in portrait or landscape layout) up to $11''\times17''$ is also permissible, but only in an appendix or supplemental.
For keeping this template simpler, we leave this larger page commented-out.
If you need this a larger page, feel free to un-comment these lines and use the example code as your template.

\begin{landscape}
\vspace*{\fill} % Use this vertical fill macro both above and below the figure/table to center it vertically on the page
\begin{figure}[H]
\centering
    \fbox{
        \includegraphics[width=0.45\linewidth]{figs/dragon.jpg}
        \includegraphics[width=0.45\linewidth]{figs/dragon.jpg}
    }
\caption{A Wide Figure}
\label{fig:wide}
\end{figure}
\vspace*{\fill}
\end{landscape}

% \newstocksize{layoutsize={11in,17in},keepmargins} % Using this requires the stocksize package, which requires TeXLive 2025 or later!
% \begin{landscape} % If you need a landscape page, keep the dimensions in the previous line portrait and use the landscape environment
% \vspace*{\fill} % Use this vertical fill macro both above and below the figure/table to center it vertically on the page
% \begin{figure}[H]
% \centering
%     \fbox{
%         \includegraphics[width=0.45\linewidth]{figs/dragon.jpg}
%         \includegraphics[width=0.45\linewidth]{figs/dragon.jpg}
%     }
% \caption{A Very Big Figure}
% \label{fig:really-big}
% \end{figure}
% \vspace*{\fill}
% \end{landscape}
% \restorestocksize

\section{Creating and Using Macros}
\LaTeX{} macros are commands that you can use as shortcuts to reproducing the same code.
For example, if you want to leave notes to yourself that contrast well with your thesis text, consider putting this macro definition in the main.tex file with your \verb|\usepackage{}| commands:

\verb|\newcommand{\jl}[1]{\textcolor{red}{JL: #1}}|

In this case, every time you put \verb|\jl{fix this!}| in your \LaTeX{} source, you will see \textcolor{red}{JL: fix this!} in your text.
The first parameter of \verb|\newcommand| is the name of the macro that you are defining.  In this case, I pretended my initials are JL.  You can use any name for a macro that is not already used.
The second parameter tells \LaTeX{} to expect one (1) parameter to this new macro.
The third parameter tells \LaTeX{} what to do when seeing this new macro.  In this case, it prints in red text: my initials, a colon, a space, and the text that I provide to the command.

If you need to have all of your comments disappear, for example, to submit your thesis draft to the TPO for its initial review, you can renew the command to hide all of your comments.  Just insert this line right after the above \verb|\newcommand| line:

\verb|\renewcommand{\jl}[1]{}|

As with nearly everything in \LaTeX{}, there is always a fancier way to do something.
In the case of making notes to remind yourself of something in a part of the text, the \texttt{todonotes} package offers a lot of features that you might find helpful.

% \chapter{[Second appendix title]}
% Sometimes it can be helpful to put extra information into an appendix.

\section{Less Common Figure Examples} \label{sec:uncommon-figures}
Here are a few example figure types that are less common.

\begin{figure}[H] % floats ask LaTeX to attempt to place figure here; used because fig originally landed on the next page, leaving big white space at the bottom of this page.
    \framebox[\textwidth]{\parbox{\textwidth}{\lipsum[65]}}
    \caption*{\small This is the long subcaption that explains the figure in detail and expounds on its relevance to the text. The other formatting option for long captions is shown in Figure \ref{fig:dragon}; choose one method for your thesis. All figures need to be referenced in the text before the image. Full source citation, as applicable, is required. Source:~\citet{Wunkerbunk:2002}.} \vspace*{1.5em}
    \caption{Figure caption, with \emph{emph} and \textit{italics} in a caption.}
\end{figure}

\begin{figure}[H] % H prevents figures from landing in the middle of paragraphs
    \framebox[\textwidth]{\parbox{\textwidth}{\lipsum[65]}}
    \caption{Some styled math in a caption, $\mathsf{Func}(x, \sigma) = x^2 + \overline{\sigma} \times \pi$.}
\end{figure}

Do you need to incorporate multiple subfigures within one figure?

\begin{figure}[H] % H prevents figures from landing in the middle of paragraphs
    \centering
    \subfloat[First sub-figure]{
       \framebox[0.47\textwidth]{\parbox{0.45\textwidth}{\lipsum[65]}}
    }
    \hfill
    \subfloat[Second sub-figure]{
       \framebox[0.47\textwidth]{\parbox{0.45\textwidth}{\lipsum[65]}}
    }
    
    \caption{Caption using subfigure package.}
\end{figure}

\subsection{A Long Code Listing}
As introduced in Section~\ref{sec:figures}, you can also use figures to display source code.
When your code listing is longer than one page, like the following example, it should have its own appendix or section in an appendix rather than being a figure.

\lstinputlisting[language=bash,numbers=left,breaklines,frame=single,aboveskip=0.6in]{figs/bash-long.sh}
% Use the `aboveskip=0.6in` parameter if you have text between the appendix title and the code listing.

\subsection{A Long Table}
As introduced in Section~\ref{section:tableformat}, you may need to show data in a table that is longer than one page, like Table~\ref{tab:long}.

\begin{center} % NB:  Unlike putting \centering within a table environment, the longtable environment goes inside the center environment
\begin{longtable}{lr}
    \caption{A sample long table showing decimal ASCII 55--90} \label{tab:long} \\ \hline
    Character & Number \\ \hline \endfirsthead % This caption and row appear at the top of the table
    \caption[]{A sample long table showing decimal ASCII 55--90} \\ \hline
    Character & Number \\ \hline \endhead % This caption and row appears at the top of each subsequent page
    % Very important:  The caption for each subsequent page must have [] to prevent the table from being listed multiple times (once for each page!) in the List of Tables.  Also, the label should appear only on the first instance of the caption.
    \hline \multicolumn{2}{r}{Continued on next page\ldots} \endfoot % This row appears at the bottom of all but the last page of the table
    \hline Total & 36 \\ \hline \endlastfoot % This row appears at the end of the table and is not always necessary.
    % You need to specify the above before providing your table data below.
    7 & 55 \\
    8 & 56 \\
    9 & 57 \\
    : & 58 \\
    ; & 59 \\
    < & 60 \\
    = & 61 \\
    > & 62 \\
    ? & 63 \\
    @ & 64 \\
    A & 65 \\
    B & 66 \\
    C & 67 \\
    D & 68 \\
    E & 69 \\
    F & 70 \\
    G & 71 \\
    H & 72 \\
    I & 73 \\
    J & 74 \\
    K & 75 \\
    L & 76 \\
    M & 77 \\
    N & 78 \\
    O & 79 \\
    P & 80 \\
    Q & 81 \\
    R & 82 \\
    S & 83 \\
    T & 84 \\
    U & 85 \\
    V & 86 \\
    W & 87 \\
    X & 88 \\
    Y & 89 \\
    Z & 90 \\
\end{longtable}
\end{center}

\subsection{A Wide Figure or Table}
If you need to present a figure or table that is too wide to fit between the margins, you can put in on a landscape page, as in Figure~\ref{fig:wide}.
Increasing the page size (in portrait or landscape layout) up to $11''\times17''$ is also permissible, but only in an appendix or supplemental.
For keeping this template simpler, we leave this larger page commented-out.
If you need this a larger page, feel free to un-comment these lines and use the example code as your template.

\begin{landscape}
\vspace*{\fill} % Use this vertical fill macro both above and below the figure/table to center it vertically on the page
\begin{figure}[H]
\centering
    \fbox{
        \includegraphics[width=0.45\linewidth]{figs/dragon.jpg}
        \includegraphics[width=0.45\linewidth]{figs/dragon.jpg}
    }
\caption{A Wide Figure}
\label{fig:wide}
\end{figure}
\vspace*{\fill}
\end{landscape}

% \newstocksize{layoutsize={11in,17in},keepmargins} % Using this requires the stocksize package, which requires TeXLive 2025 or later!
% \begin{landscape} % If you need a landscape page, keep the dimensions in the previous line portrait and use the landscape environment
% \vspace*{\fill} % Use this vertical fill macro both above and below the figure/table to center it vertically on the page
% \begin{figure}[H]
% \centering
%     \fbox{
%         \includegraphics[width=0.45\linewidth]{figs/dragon.jpg}
%         \includegraphics[width=0.45\linewidth]{figs/dragon.jpg}
%     }
% \caption{A Very Big Figure}
% \label{fig:really-big}
% \end{figure}
% \vspace*{\fill}
% \end{landscape}
% \restorestocksize

\section{Creating and Using Macros}
\LaTeX{} macros are commands that you can use as shortcuts to reproducing the same code.
For example, if you want to leave notes to yourself that contrast well with your thesis text, consider putting this macro definition in the main.tex file with your \verb|\usepackage{}| commands:

\verb|\newcommand{\jl}[1]{\textcolor{red}{JL: #1}}|

In this case, every time you put \verb|\jl{fix this!}| in your \LaTeX{} source, you will see \textcolor{red}{JL: fix this!} in your text.
The first parameter of \verb|\newcommand| is the name of the macro that you are defining.  In this case, I pretended my initials are JL.  You can use any name for a macro that is not already used.
The second parameter tells \LaTeX{} to expect one (1) parameter to this new macro.
The third parameter tells \LaTeX{} what to do when seeing this new macro.  In this case, it prints in red text: my initials, a colon, a space, and the text that I provide to the command.

If you need to have all of your comments disappear, for example, to submit your thesis draft to the TPO for its initial review, you can renew the command to hide all of your comments.  Just insert this line right after the above \verb|\newcommand| line:

\verb|\renewcommand{\jl}[1]{}|

As with nearly everything in \LaTeX{}, there is always a fancier way to do something.
In the case of making notes to remind yourself of something in a part of the text, the \texttt{todonotes} package offers a lot of features that you might find helpful.



% SUPPLEMENTALS
% If you want to include a supplemental with your thesis, Follow the instructions at https://nps.edu/web/thesisprocessing/supplementals.
% The file linked in the following line shows an example.
% % Follow the instructions (https://nps.edu/web/thesisprocessing/supplementals) for the information on this page.
% This file includes two examples.  The first example is for one supplemental.  The second example is for two supplementals.
% Un-comment whichever example you would like to use.

%%%%%%%%%%%%%%%%%%%%%%%%%
% For a supplemental, you can identify it here, including a brief description.
%%%%%%%%%%%%%%%%%%%%%%%%%

% \chapter*{Supplemental:  Collected Query Data} \label{supp} % This label provides a way to cross-reference this supplemental, like a Figure, Equation, Section, Appendix, etc.
% \addcontentsline{toc}{chapter}{Supplemental:  Collected Query Data}

% Modify the chapter title above and this paragraph:  Include a paragraph or two describing the supplemental material.  Remain vague about the supplemental's ultimate format.

% To access the supplemental material described here, contact the \href{https://libanswers.nps.edu/}{Dudley Knox Library} (\url{https://libanswers.nps.edu/}) or, for publicly releasable theses and supplementals only, visit the thesis pages in the \href{https://library.nps.edu/nps-theses}{library's Calhoun database} (\url{https://library.nps.edu/nps-theses}).



%%%%%%%%%%%%%%%%%%%%%%%%%
% For multiple supplementals, you can list them here, including a brief description of each.
%%%%%%%%%%%%%%%%%%%%%%%%%

% \chapter*{Supplementals}
% \addcontentsline{toc}{chapter}{Supplementals}

% To access the supplemental material(s) listed here, contact the \href{https://libanswers.nps.edu/}{Dudley Knox Library} (\url{https://libanswers.nps.edu/}) or, for publicly releasable theses and supplementals only, visit the thesis pages in the \href{https://library.nps.edu/nps-theses}{library's Calhoun database} (\url{https://library.nps.edu/nps-theses}).

% \newcounter{supp}
% \newcommand{\totsupp}{2} % Enter the total number of supplementals here.

% \refstepcounter{supp} \label{supp:querydata} % This label provides a way to cross-reference this supplemental, like a Figure, Equation, Section, Appendix, etc.
% Supplemental \#\thesupp{} of \totsupp:  Collected Query Data (Unclassified):  Brief description

% \refstepcounter{supp} \label{supp:systemDescription}
% Supplemental \#\thesupp{} of \totsupp:  A Description of the Studied System (CUI):  Brief description



% REFERENCES
% List all your BibTeX reference source files (ending in *.bib extension)

% This command builds the bibliography:
% 1. The default thesis style is nps-ieee.bst for IEEE and nps-informs.bst for INFORMS
% 2. The argument "references" tells LaTeX to find your bib items in the file references.bib
\NPSbibliography{references}

% Notes:
% 1. You can include multiple bib files (just comma separate them), for example:  \NPSbibliography{references,other-refs}
% 2. Ensure the formatting matches the library's guidance (https://libguides.nps.edu/citation/bibtex).
% 3. If not using a bib style (.bst) offered in the template, you can specify it like this:  \NPSbibliography[chicago]{references}


%%%%%%%%%%%%%%%%%%%%%%%%%
% This is the official end of the thesis.
%%%%%%%%%%%%%%%%%%%%%%%%%
\NPSend

% DISTRIBUTION LIST
% The list is automatically properly numbered
% and already populated with the mandatory recipients.
%
\NPSdistribution{Initial Distribution List}
\begin{distributionlist}
\item Defense Technical Information Center\\Fort Belvoir, Virginia
\item Dudley Knox Library\\Naval Postgraduate School\\Monterey, California
%

\end{distributionlist}

\end{document}
